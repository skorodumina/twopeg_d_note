\noindent\begin{minipage}{\textwidth}
\begin{center}
\thispagestyle{empty}
\vspace{0.5cm}
{ \Large{TWOPEG-D: An Extension of TWOPEG for the Case of\\ a Moving Proton Target}}\\
\vspace{1cm}

{\large Iu. Skorodumina$^{1, a}$, G.V. Fedotov$^{2, 3, b}$,  R.W. Gothe$^{1}$} \\[16pt]

\parbox{.86\textwidth}{\centering\footnotesize\it
$^1$ Department of Physics and Astronomy, University of South Carolina, Columbia, SC 29208.\\
$^2$ Ohio University, Athens, Ohio  45701\\
$^3$ Skobel'tsyn Institute of Nuclear Physics, Moscow State University, Moscow, 119991, Russia\\
E-mail: $^a$ skorodum@jlab.org, $^b$ gleb@jlab.org}\\


\vspace{2cm}
{\bf Abstract}\\[9pt]

\end{center}
{A new TWOPEG-D version of the event generator TWOPEG was developed. This new version simulates the quasi-free process of double-pion electroproduction off the proton that moves in the deuteron target. The underlying idea is the equivalence of the moving proton experiment performed with fixed laboratory beam energy to the proton at rest experiment conducted with effective beam energy different from the laboratory one. This effective beam energy differs event by event and is determined by the boost from the Lab system to the proton rest frame. The Fermi momentum of the target proton is generated according to the Bonn potential. The specific aspects of the deuteron target data analysis are discussed. The plots that illustrate the performance of TWOPEG-D are given. The link to the code is provided. The generator was tested in the analysis of the CLAS data on electron scattering off the deuteron target.}

%\vspace{1pt}\par
%equivalence of the electron scattering experiment performed with a fixed laboratory beam energy on a moving proton to that on the proton at rest conducted with the effective beam energy different from the laboratory one.

\end{minipage}
